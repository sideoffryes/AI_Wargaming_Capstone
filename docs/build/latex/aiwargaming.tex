%% Generated by Sphinx.
\def\sphinxdocclass{report}
\documentclass[letterpaper,10pt,english]{sphinxmanual}
\ifdefined\pdfpxdimen
   \let\sphinxpxdimen\pdfpxdimen\else\newdimen\sphinxpxdimen
\fi \sphinxpxdimen=.75bp\relax
\ifdefined\pdfimageresolution
    \pdfimageresolution= \numexpr \dimexpr1in\relax/\sphinxpxdimen\relax
\fi
%% let collapsible pdf bookmarks panel have high depth per default
\PassOptionsToPackage{bookmarksdepth=5}{hyperref}

\PassOptionsToPackage{booktabs}{sphinx}
\PassOptionsToPackage{colorrows}{sphinx}

\PassOptionsToPackage{warn}{textcomp}
\usepackage[utf8]{inputenc}
\ifdefined\DeclareUnicodeCharacter
% support both utf8 and utf8x syntaxes
  \ifdefined\DeclareUnicodeCharacterAsOptional
    \def\sphinxDUC#1{\DeclareUnicodeCharacter{"#1}}
  \else
    \let\sphinxDUC\DeclareUnicodeCharacter
  \fi
  \sphinxDUC{00A0}{\nobreakspace}
  \sphinxDUC{2500}{\sphinxunichar{2500}}
  \sphinxDUC{2502}{\sphinxunichar{2502}}
  \sphinxDUC{2514}{\sphinxunichar{2514}}
  \sphinxDUC{251C}{\sphinxunichar{251C}}
  \sphinxDUC{2572}{\textbackslash}
\fi
\usepackage{cmap}
\usepackage[T1]{fontenc}
\usepackage{amsmath,amssymb,amstext}
\usepackage{babel}



\usepackage{tgtermes}
\usepackage{tgheros}
\renewcommand{\ttdefault}{txtt}



\usepackage[Bjarne]{fncychap}
\usepackage{sphinx}

\fvset{fontsize=auto}
\usepackage{geometry}


% Include hyperref last.
\usepackage{hyperref}
% Fix anchor placement for figures with captions.
\usepackage{hypcap}% it must be loaded after hyperref.
% Set up styles of URL: it should be placed after hyperref.
\urlstyle{same}

\addto\captionsenglish{\renewcommand{\contentsname}{Contents:}}

\usepackage{sphinxmessages}
\setcounter{tocdepth}{1}



\title{AI Wargaming}
\date{Feb 13, 2025}
\release{0.3}
\author{Peter Asjes, Henry Frye, Caleb Koutrakos, Will Robinson, and Bobby Ziman}
\newcommand{\sphinxlogo}{\vbox{}}
\renewcommand{\releasename}{Release}
\makeindex
\begin{document}

\ifdefined\shorthandoff
  \ifnum\catcode`\=\string=\active\shorthandoff{=}\fi
  \ifnum\catcode`\"=\active\shorthandoff{"}\fi
\fi

\pagestyle{empty}
\sphinxmaketitle
\pagestyle{plain}
\sphinxtableofcontents
\pagestyle{normal}
\phantomsection\label{\detokenize{index::doc}}


\sphinxAtStartPar
\sphinxstylestrong{AI Wargaming} is a undergraduate capstone project part of a larger initiative to modernize wargaming using AI.

\begin{sphinxadmonition}{note}{Note:}
\sphinxAtStartPar
This project is under active development.
\end{sphinxadmonition}

\sphinxstepscope


\chapter{Usage}
\label{\detokenize{usage:usage}}\label{\detokenize{usage::doc}}

\section{Installation}
\label{\detokenize{usage:installation}}

\subsection{Download the Code}
\label{\detokenize{usage:download-the-code}}
\sphinxAtStartPar
The code for this project can be downloaded via \sphinxhref{https://github.com/sideoffryes/AI\_Wargaming\_Capstone/tree/main}{GitHub}.

\begin{sphinxVerbatim}[commandchars=\\\{\}]
\PYG{g+gp}{\PYGZdl{} }git\PYG{+w}{ }clone\PYG{+w}{ }https://github.com/sideoffryes/AI\PYGZus{}Wargaming\PYGZus{}Capstone.git
\end{sphinxVerbatim}


\subsection{Conda and Python Management}
\label{\detokenize{usage:conda-and-python-management}}
\sphinxAtStartPar
To use this project, the are several prerequisites that are necessary. The easiest way to manage these dependencies is using Conda.

\sphinxAtStartPar
If you do not have Conda already, \sphinxhref{https://docs.anaconda.com/miniconda/install/}{download and install a release} for your OS.

\sphinxAtStartPar
All of the required python packages can be easily installed via the provided configuration files. There are separate files for GPU and CPU dependencies.


\subsubsection{CPU}
\label{\detokenize{usage:cpu}}
\begin{sphinxVerbatim}[commandchars=\\\{\}]
\PYG{g+gp+gpVirtualEnv}{(base)} \PYG{g+gp}{\PYGZdl{} }conda\PYG{+w}{ }env\PYG{+w}{ }create\PYG{+w}{ }\PYGZhy{}f\PYG{+w}{ }environment\PYGZus{}cpu.yml
\end{sphinxVerbatim}


\subsubsection{GPU}
\label{\detokenize{usage:gpu}}
\begin{sphinxVerbatim}[commandchars=\\\{\}]
\PYG{g+gp+gpVirtualEnv}{(base)} \PYG{g+gp}{\PYGZdl{} }conda\PYG{+w}{ }env\PYG{+w}{ }create\PYG{+w}{ }\PYGZhy{}f\PYG{+w}{ }environment\PYGZus{}gpu.yml
\end{sphinxVerbatim}


\subsection{Hugging Face}
\label{\detokenize{usage:hugging-face}}
\sphinxAtStartPar
The project accesses the Llama 3.1, 3.2, and 3.3 families from Meta via Hugging Face. Running this project requires a Hugging Face account and access to those families.
\begin{enumerate}
\sphinxsetlistlabels{\arabic}{enumi}{enumii}{}{.}%
\item {} 
\sphinxAtStartPar
Visit the page for the model on Hugging Face. For example, \sphinxhref{https://huggingface.co/meta-llama/Llama-3.2-1B}{Llama\sphinxhyphen{}3.2 (1B)}

\item {} 
\sphinxAtStartPar
Create a free account and login.

\item {} 
\sphinxAtStartPar
Return to the Llama webpage (if not already there).

\item {} 
\sphinxAtStartPar
You should see a Community License Agreement at the top. Click the “Expand to review” button:

\item {} 
\sphinxAtStartPar
If you agree with the terms, fill out the form

\item {} 
\sphinxAtStartPar
Check email later.

\end{enumerate}

\sphinxAtStartPar
Once you have received access to the models, visit your \sphinxhref{https://huggingface.co/settings/tokens}{tokens page} and click “Create new token”. Choose the “Read” token type at the very top. Then click “Create token”. Copy the generated string.

\sphinxAtStartPar
In the terminal, run the following command and paste in your access token when prompted:

\begin{sphinxVerbatim}[commandchars=\\\{\}]
\PYG{g+gp+gpVirtualEnv}{(base)} \PYG{g+gp}{\PYGZdl{} }huggingface\PYGZhy{}cli\PYG{+w}{ }login
\end{sphinxVerbatim}


\section{Running the Project}
\label{\detokenize{usage:running-the-project}}
\sphinxAtStartPar
Before attempting to run any of the scripts, make sure that you have the correct Conda environment activated.

\begin{sphinxVerbatim}[commandchars=\\\{\}]
\PYG{g+gp+gpVirtualEnv}{(base)} \PYG{g+gp}{\PYGZdl{} }conda\PYG{+w}{ }activate\PYG{+w}{ }capstone\PYGZus{}gpu
\end{sphinxVerbatim}

\sphinxAtStartPar
All of the code that runs the webserver and actually generated the documents can be found inside of the \sphinxstyleemphasis{capstone} directory. \sphinxstyleemphasis{app.py} is the webserver and \sphinxstyleemphasis{docgen.py} is the script that accesses the LLM to generate documents. Running the entire project can be accomplished with the following:

\begin{sphinxVerbatim}[commandchars=\\\{\}]
\PYG{g+gp+gpVirtualEnv}{(capstone\PYGZus{}gpu)} \PYG{g+gp}{\PYGZdl{} }\PYG{n+nb}{cd}\PYG{+w}{ }capstone
\PYG{g+gp+gpVirtualEnv}{(capstone\PYGZus{}gpu)} \PYG{g+gp}{\PYGZdl{} }python3\PYG{+w}{ }app.py
\end{sphinxVerbatim}

\sphinxAtStartPar
The webserver can be reached from your browser by using one of the ip addresses printed out in the terminal when the server is created.


\section{Interacting with the Webserver in the Browser}
\label{\detokenize{usage:interacting-with-the-webserver-in-the-browser}}
\sphinxAtStartPar
The form presented to you when the website is first loaded can be used to generate a document. Use the \sphinxstyleemphasis{selection options} dropdown menu to select the type of document that you would like to create. You can specify your requirements and any additional specifications in the \sphinxstyleemphasis{additional parameters} textbox.

\sphinxAtStartPar
Depending on the size of the model used to generate the document, the server may load for a few minutes before the final output is produced.

\sphinxstepscope


\chapter{API}
\label{\detokenize{api:api}}\label{\detokenize{api::doc}}

\begin{savenotes}\sphinxattablestart
\sphinxthistablewithglobalstyle
\sphinxthistablewithnovlinesstyle
\centering
\begin{tabulary}{\linewidth}[t]{\X{1}{2}\X{1}{2}}
\sphinxtoprule
\sphinxtableatstartofbodyhook
\sphinxAtStartPar
{\hyperref[\detokenize{generated/docGen:module-docGen}]{\sphinxcrossref{\sphinxcode{\sphinxupquote{docGen}}}}}
&
\sphinxAtStartPar

\\
\sphinxhline
\sphinxAtStartPar
{\hyperref[\detokenize{generated/app:module-app}]{\sphinxcrossref{\sphinxcode{\sphinxupquote{app}}}}}
&
\sphinxAtStartPar

\\
\sphinxhline
\sphinxAtStartPar
{\hyperref[\detokenize{generated/pdfToText:module-pdfToText}]{\sphinxcrossref{\sphinxcode{\sphinxupquote{pdfToText}}}}}
&
\sphinxAtStartPar

\\
\sphinxbottomrule
\end{tabulary}
\sphinxtableafterendhook\par
\sphinxattableend\end{savenotes}

\sphinxstepscope


\section{docGen}
\label{\detokenize{generated/docGen:module-docGen}}\label{\detokenize{generated/docGen:docgen}}\label{\detokenize{generated/docGen::doc}}\index{module@\spxentry{module}!docGen@\spxentry{docGen}}\index{docGen@\spxentry{docGen}!module@\spxentry{module}}\subsubsection*{Functions}


\begin{savenotes}\sphinxattablestart
\sphinxthistablewithglobalstyle
\sphinxthistablewithnovlinesstyle
\centering
\begin{tabulary}{\linewidth}[t]{\X{1}{2}\X{1}{2}}
\sphinxtoprule
\sphinxtableatstartofbodyhook
\sphinxAtStartPar
\sphinxcode{\sphinxupquote{gen}}(model\_num, type\_num, prompt{[}, save{]})
&
\sphinxAtStartPar
Generates a specificed document using a specified LLM and returns the result.
\\
\sphinxhline
\sphinxAtStartPar
\sphinxcode{\sphinxupquote{load\_examples}}(type)
&
\sphinxAtStartPar
Returns real life examples of the requested document type.
\\
\sphinxhline
\sphinxAtStartPar
\sphinxcode{\sphinxupquote{save\_response}}(response, prompt, model\_name, ...)
&
\sphinxAtStartPar
Writes the response and information about how it was generated to the disk.
\\
\sphinxhline
\sphinxAtStartPar
\sphinxcode{\sphinxupquote{select\_doc}}(num)
&
\sphinxAtStartPar
Translates between the numeric representation of a document type and its full name.
\\
\sphinxhline
\sphinxAtStartPar
\sphinxcode{\sphinxupquote{select\_model}}(num)
&
\sphinxAtStartPar
Translates between the numeric representation of a model to the full string of its name.
\\
\sphinxbottomrule
\end{tabulary}
\sphinxtableafterendhook\par
\sphinxattableend\end{savenotes}

\sphinxstepscope


\section{app}
\label{\detokenize{generated/app:module-app}}\label{\detokenize{generated/app:app}}\label{\detokenize{generated/app::doc}}\index{module@\spxentry{module}!app@\spxentry{app}}\index{app@\spxentry{app}!module@\spxentry{module}}\subsubsection*{Functions}


\begin{savenotes}\sphinxattablestart
\sphinxthistablewithglobalstyle
\sphinxthistablewithnovlinesstyle
\centering
\begin{tabulary}{\linewidth}[t]{\X{1}{2}\X{1}{2}}
\sphinxtoprule
\sphinxtableatstartofbodyhook
\sphinxAtStartPar
\sphinxcode{\sphinxupquote{createAcc}}()
&
\sphinxAtStartPar

\\
\sphinxhline
\sphinxAtStartPar
\sphinxcode{\sphinxupquote{docs}}(filename)
&
\sphinxAtStartPar

\\
\sphinxhline
\sphinxAtStartPar
\sphinxcode{\sphinxupquote{handle\_indexPost}}()
&
\sphinxAtStartPar

\\
\sphinxhline
\sphinxAtStartPar
\sphinxcode{\sphinxupquote{handle\_loginPost}}()
&
\sphinxAtStartPar

\\
\sphinxhline
\sphinxAtStartPar
\sphinxcode{\sphinxupquote{handle\_registerPost}}()
&
\sphinxAtStartPar

\\
\sphinxhline
\sphinxAtStartPar
\sphinxcode{\sphinxupquote{home}}()
&
\sphinxAtStartPar

\\
\sphinxhline
\sphinxAtStartPar
\sphinxcode{\sphinxupquote{index}}()
&
\sphinxAtStartPar

\\
\sphinxhline
\sphinxAtStartPar
\sphinxcode{\sphinxupquote{login}}()
&
\sphinxAtStartPar

\\
\sphinxhline
\sphinxAtStartPar
\sphinxcode{\sphinxupquote{logout}}()
&
\sphinxAtStartPar

\\
\sphinxhline
\sphinxAtStartPar
\sphinxcode{\sphinxupquote{my\_artifacts}}()
&
\sphinxAtStartPar

\\
\sphinxhline
\sphinxAtStartPar
\sphinxcode{\sphinxupquote{profile}}()
&
\sphinxAtStartPar

\\
\sphinxbottomrule
\end{tabulary}
\sphinxtableafterendhook\par
\sphinxattableend\end{savenotes}
\subsubsection*{Classes}


\begin{savenotes}\sphinxattablestart
\sphinxthistablewithglobalstyle
\sphinxthistablewithnovlinesstyle
\centering
\begin{tabulary}{\linewidth}[t]{\X{1}{2}\X{1}{2}}
\sphinxtoprule
\sphinxtableatstartofbodyhook
\sphinxAtStartPar
\sphinxcode{\sphinxupquote{GeneratedArtifact}}(**kwargs)
&
\sphinxAtStartPar

\\
\sphinxhline
\sphinxAtStartPar
\sphinxcode{\sphinxupquote{Profile}}(**kwargs)
&
\sphinxAtStartPar

\\
\sphinxbottomrule
\end{tabulary}
\sphinxtableafterendhook\par
\sphinxattableend\end{savenotes}

\sphinxstepscope


\section{pdfToText}
\label{\detokenize{generated/pdfToText:module-pdfToText}}\label{\detokenize{generated/pdfToText:pdftotext}}\label{\detokenize{generated/pdfToText::doc}}\index{module@\spxentry{module}!pdfToText@\spxentry{pdfToText}}\index{pdfToText@\spxentry{pdfToText}!module@\spxentry{module}}\subsubsection*{Functions}


\begin{savenotes}\sphinxattablestart
\sphinxthistablewithglobalstyle
\sphinxthistablewithnovlinesstyle
\centering
\begin{tabulary}{\linewidth}[t]{\X{1}{2}\X{1}{2}}
\sphinxtoprule
\sphinxtableatstartofbodyhook
\sphinxAtStartPar
\sphinxcode{\sphinxupquote{pdf\_to\_text}}(pdf\_path, output\_txt)
&
\sphinxAtStartPar

\\
\sphinxbottomrule
\end{tabulary}
\sphinxtableafterendhook\par
\sphinxattableend\end{savenotes}


\chapter{Indices and tables}
\label{\detokenize{index:indices-and-tables}}\begin{itemize}
\item {} 
\sphinxAtStartPar
\DUrole{xref}{\DUrole{std}{\DUrole{std-ref}{genindex}}}

\item {} 
\sphinxAtStartPar
\DUrole{xref}{\DUrole{std}{\DUrole{std-ref}{modindex}}}

\item {} 
\sphinxAtStartPar
\DUrole{xref}{\DUrole{std}{\DUrole{std-ref}{search}}}

\end{itemize}


\renewcommand{\indexname}{Python Module Index}
\begin{sphinxtheindex}
\let\bigletter\sphinxstyleindexlettergroup
\bigletter{a}
\item\relax\sphinxstyleindexentry{app}\sphinxstyleindexpageref{generated/app:\detokenize{module-app}}
\indexspace
\bigletter{d}
\item\relax\sphinxstyleindexentry{docGen}\sphinxstyleindexpageref{generated/docGen:\detokenize{module-docGen}}
\indexspace
\bigletter{p}
\item\relax\sphinxstyleindexentry{pdfToText}\sphinxstyleindexpageref{generated/pdfToText:\detokenize{module-pdfToText}}
\end{sphinxtheindex}

\renewcommand{\indexname}{Index}
\printindex
\end{document}